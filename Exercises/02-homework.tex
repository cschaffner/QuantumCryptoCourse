\documentclass[a4paper,10pt,landscape,twocolumn]{scrartcl}

\newcommand{\ket}[1]{| #1 \rangle}
\newcommand{\bra}[1]{\langle #1 |}
\newcommand{\proj}[1]{| #1 \rangle \langle #1 |}
\newcommand{\Tr}{\text{Tr}}

%% Settings
\newcommand\problemset{2}
\newcommand\deadline{Friday June 23, 20:00h}
\newif\ifcomments
\commentsfalse % hide comments
%\commentstrue % show comments

% Packages
\usepackage[english]{exercises}
\usepackage{wasysym}
\usepackage{hyperref}
\usepackage{mathrsfs} 
\hypersetup{colorlinks=true, urlcolor = blue, linkcolor = blue}
\usepackage{bbm}
\begin{document}

\newcommand{\Hmin}{\mathrm{H}_{\mathrm{min}}}

\homeworkproblems

{\sffamily\noindent
%This week's exercises deal with sets, counting and uniform probabilities.
Please hand in your solutions to these exercises in digital form (typed, or scanned from a neatly hand-written version) through Moodle no later than \textbf{\deadline}.  %Problems marked with a $\bigstar$ are generally a bit harder.
}

\begin{exercise}[Min-Entropy Chain rule for cq-states]
Let $\rho_{XE} = \sum_x P_X(x) \ket{x}\bra{x} \otimes \rho_E^x$ be a cq-state. Prove the following chain rule:
\[
\Hmin(X | E) \geq \Hmin(X) - \log |E| \, .
\]
\textbf{Hint: } Use the fact that $0 \leq \rho_E^x \leq \mathbbm{1}$.

\end{exercise}

\begin{exercise}[injective functions are collapsing]
Show that an injective function is collapsing, i.e. give a proof of Lemma~2 of \href{http://homepages.cwi.nl/~schaffne/spool/sponges.pdf}{our recent paper}. You can ignore the oracles $\mathcal{O}$ in the statement of Lemma 2 and in Definition~1.
\end{exercise}

\begin{exercise}[A weak seeded extractor]
	For any $y \in \{0,1\}^n$, define $f_y : \{0,1\}^n \to \{0,1\}^n$ by $f_y(x) = x \oplus y$. Here, $\oplus$ represents the bitwise parity (e.g., $11 \oplus 01 = 10$).
	\begin{subex}
		Show that the family $\mathscr{F} = \{f_y\}$ is 1-universal.
	\end{subex}
        \begin{subex}
          Show that the family $\mathscr{F} = \{f_y\}$ is not 2-universal.
        \end{subex}
        \begin{subex}
    	How could you use $\mathscr{F}$ to build a $(k,0)$-weak seeded randomness extractor $\text{Ext} : \{0,1\}^n \times \{0,1\}^n \to \{0,1\}^n$ for any $k$. Is this extractor useful?
    \end{subex}
    \begin{subex}
    	Alice and Bob are impressed by the parameter $\epsilon = 0$ in the previous exercise. They decide that if $\mathscr{F}$ can be used for a $(k,0)$-weak seeded randomness extractor, then certainly it can reasonably be used as a \textbf{strong} seeded randomness extractor as well. They define $\text{Ext}(x,y) = f_y(x)$. Do you think this is a good idea? How does Eve's guessing probability change after extraction?
    \end{subex}
\end{exercise}



\end{document}