\documentclass[a4paper,10pt,landscape,twocolumn]{scrartcl}

\newcommand{\ket}[1]{| #1 \rangle}
\newcommand{\bra}[1]{\langle #1 |}
\newcommand{\proj}[1]{| #1 \rangle \langle #1 |}
\newcommand{\Tr}{\text{Tr}}

%% Settings
\newcommand\problemset{2}
\newcommand\deadline{Friday June 23, 20:00h}
\newif\ifcomments
\commentsfalse % hide comments
%\commentstrue % show comments

% Packages
\usepackage[english]{exercises}
\usepackage{wasysym}
\usepackage{hyperref}
\usepackage{mathrsfs} 
\hypersetup{colorlinks=true, urlcolor = blue, linkcolor = blue}
\usepackage{bbm}
\begin{document}

\newcommand{\Hmin}{\mathrm{H}_{\mathrm{min}}}

\homeworkproblems

{\sffamily\noindent
%This week's exercises deal with sets, counting and uniform probabilities.
Please hand in your solutions to these exercises in digital form (typed, or scanned from a neatly hand-written version) through Canvas no later than \textbf{\deadline}.  %Problems marked with a $\bigstar$ are generally a bit harder.
}
\begin{exercise}[Injective functions are collapsing]
Show that an injective function is collapsing, i.e. give a proof of Lemma~4 of \href{https://eprint.iacr.org/2017/771.pdf}{our recent paper}. You can ignore the oracles $\mathcal{O}$ in the statement of Lemma~4 and in Definition~3.
\end{exercise}


\begin{exercise}[A weak seeded extractor]
	For any $y \in \{0,1\}^n$, define $f_y : \{0,1\}^n \to \{0,1\}^n$ by $f_y(x) = x \oplus y$. Here, $\oplus$ represents the bitwise parity (e.g., $11 \oplus 01 = 10$).
	\begin{subex}
		Show that the family $\mathscr{F} = \{f_y\}$ is 1-universal.
	\end{subex}
    \begin{subex}
        Show that the family $\mathscr{F} = \{f_y\}$ is not 2-universal.
    \end{subex}
    \begin{subex}
    	How could you use $\mathscr{F}$ to build a $(k,0)$-weak seeded randomness extractor $\text{Ext} : \{0,1\}^n \times \{0,1\}^n \to \{0,1\}^n$, for any $0 \leq k \leq n$? Is this extractor useful?
    \end{subex}
    \begin{subex}
    	Alice and Bob have a $k$-source $X$ for some $k < n$. They are impressed by the parameters in the previous subexercise, and decide to use $\mathscr{F}$ to build a strong seeded randomness extractor as well. They know they should not expect to securely extract more than $k$ bits of key, so they define $\text{Ext}(x,y) := (x_1 \oplus y_1, ..., x_k \oplus y_k)$, that is, the first $k$ bits of $f_y(x)$. (From the exercise session, they know how to show that this set of functions is still 1-universal). Do you think this is a good idea? Give a lower bound to Eve's probability of guessing the key.
    \end{subex}
\end{exercise}

\begin{exercise}[Min-Entropy Chain rule for cq-states]
Let $\rho_{XE} = \sum_x P_X(x) \ket{x}\bra{x} \otimes \rho_E^x$ be a cq-state. Prove the following chain rule:
\[
\Hmin(X | E) \geq \Hmin(X) - \log |E| \, .
\]
\textbf{Hint: } Use the fact that $0 \leq \rho_E^x \leq \mathbbm{1}$.

\end{exercise}

\begin{exercise}[SARG04 Quantum Key Distribution Protocol]
A protocol propose in 2004 is a seemingly innocent variation to BB84 protocol, but has some advantages in realistic implementation
in QKD. A again sends randomly one of four states used in BB84, and B measures randomly in either horizontal-vertical or diagonal basis. However, instead of revealing the basis at the sifting stage, A announces publicly one of four pairs $\{ \ket{0},\ket{+} \}$, $\{ \ket{0},\ket{-} \}$, $\{ \ket{1},\ket{+} \}$, $\{ \ket{1},\ket{-} \}$. The announced pair contains a state send by A but it is not revealed which one. The convention is that $\ket{+}$, $\ket{-}$ are assigned logical value 0 while $\ket{0}$, $\ket{1}$ logical value 1. To understand how secret key can be generated consider situation in which A sends $\ket{1}$ and announces the pair $\{ \ket{1},\ket{+} \}$, with
probability $1/2$ B measures in the computational basis and he gets $\ket{1}$. He is not sure, however which state from the announced pair caused this results so he discard it. With probability $1/2$ he measures in the Hadamard basis in which case half of the times he gets $\ket{+}$ and half of the time he gets $\ket{-}$. Only in this last case he is sure that the state send by A was $\ket{1}$ and he writes down the bit value 0. What portion of the bits is discarded. Analyze the security of the protocol under random basis attacks. If the pairs of states had been announced before sending the qubit, could E perform a more powerful intercept and resend attack?
\end{exercise}

\begin{exercise}[Intermediate basis attack]
Analyse the intercept and resend attack in BB48 protocol in which Eve measurems in the basis intermediate between the computational and Hadamard bases.
\end{exercise}



\end{document}