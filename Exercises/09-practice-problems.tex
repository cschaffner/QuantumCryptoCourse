\documentclass[a4paper,10pt,landscape,twocolumn]{scrartcl}

\newcommand{\ket}[1]{| #1 \rangle}
\newcommand{\bra}[1]{\langle #1 |}
\newcommand{\proj}[1]{| #1 \rangle \langle #1 |}
\newcommand{\Tr}{\text{Tr}}

%% Settings
\newcommand\problemset{9}
\newcommand\deadline{Friday November 18th, 20:00h}
\newif\ifcomments
\commentsfalse % hide comments
%\commentstrue % show comments

% Packages
\usepackage[english]{exercises}
\usepackage{wasysym}
\usepackage{hyperref}
\hypersetup{colorlinks=true, urlcolor = blue, linkcolor = blue}
\usepackage{bbm}
\begin{document}

\newcommand{\Hi}{\mathcal{H}}

\practiceproblems

{\sffamily\noindent
%This week's exercises deal with sets, counting and uniform probabilities.
You do not have to hand in these exercises, they are for practicing only. %Problems marked with a $\bigstar$ are generally a bit harder.
}

\begin{exercise}[Quantum Bit Commitment (Problem 1 from Section 9.1)]
Consider the following protocol for a bit commitment: Alice prepares $\ket{\psi_0}|ψ0⟩_{AB}=\frac{1}{\sqrt{2}} ( \ket{00} + \ket{11})$ if she commits to $x=0$ or she prepares $\ket{\psi_1}_{AB} = \frac{1}{\sqrt{2}}  ( \ket{00} - \ket{11})$, if she commits to $x=1$. Then she sends the register $B$ to Bob. Finally, in the open phase, Alice just sends Bob her register $A$, so that Bob can perform a measurement in the Bell basis on the two qubits in registers $A$ and $B$ to learn Alice's bit. 
Is this protocol correct, hiding and binding?
\end{exercise}

\begin{exercise}[OT in the noisy-storage model with unbounded Alice, Problem 1 in 9.3]
In the described protocol for Oblivious Transfer in the limited-quantum-storage setting, honest Alice and Bob need no quantum memory, that is the quantum memory would only be convenient for the adversary that is trying to break the protocol. Does that mean that if Alice possessed an unbounded and noise-free quantum memory, then the protocol would not be secure against cheating Alice anymore?
\end{exercise}

\begin{exercise}[WSE without quantum memory, Problem 1 in 9.5]
Consider a quantum implementation of the weak-string erasure, where Bob has no quantum memory. For a given qubit from Alice, what is his optimal probability of guessing Alice's encoded bit, if he is allowed to measure in any basis?
\end{exercise}

\begin{exercise}[Fun with PR boxes (Homework Problems 1-5 of Chapter 9)]
In this homework, we will investigate side stepping the impossibility proof in some fun other ways! Similar to the setting of the wizard who gave Alice and Bob special forms of randomness, we'll imagine that they are given some very special resources that they can use during their protocol. We start our investigations with Alice and Bob having access to a very special box - also known as a PR-box or non-local box - used in the study of quantum non-locality. 

Imagine thus that Alice and Bob get access to a special box. This box takes two inputs: one input bit $x$ from Alice, and one input bit $y$ from Bob. Once any input is given, the box generates a random bit $r$ with $P_R(0)=P_R(1)=1/2$. The box then outputs $b=r+ x \cdot y \mod 2$ to Bob and $a=r$ to Alice.

\begin{subex}
Alice and Bob now want to find a protocol to solve the following task: Alice holds a database of two bits $x_0$ and $x_1$ and Bob holds some bit $y$. Bob would like to retrieve the bit $x_y$ from Alice's database. To achieve this, they can use the box above and in addition Alice is allowed to send one bit to Bob. Bob is not allowed to send anything to Alice. How can they do this?

\textbf{Hint: } Alice inputs $x_0 \oplus x_1$ into the PR-box, Bob inputs $y$.
\end{subex}

\begin{subex}
How well can Alice and Bob play the CHSH game in case they have a single PR box at their disposal?
\end{subex}

\begin{subex}
Slightly shocked by this outcome, Alice and Bob begin to suspect that their box is cheating. Namely they suspect their box of actually violating the non-signaling condition (i.e.\ it somehow magically allows Alice and Bob to communicate when playing the CHSH game). Therefore they resolve to test what are called the \emph{non-signaling conditions}. 

The non-signaling conditions intuitively say that if Alice and Bob input something into the box and receive an output, Alice's output should not depend on Bob's input and Bob's output should not depend on Alice's input. Formally what they will do is the following: Alice generates a bit $x$ at random such that $P_X(0)=P_X(1)=\frac12$ and inputs it in the box on her side and Bob generates a bit $y$ at random such that $P_Y(0)=P_Y(1)=\frac12$ and inputs it on his side. Then they receive output bits $a$ and $b$ from the box (distributed as described above) which they will use to check the following conditions:
\begin{align*}
\forall a,x,y,\tilde{y} \in \{0,1\}: \sum_{b \in \{0,1\}} P(a,b|x,y)= \sum_{b \in \{0,1\}} P(a,b|x,\tilde{y}), \\
\forall a,y,x,\tilde{x} \in \{0,1\}: \sum_{a \in \{0,1\}} P(a,b|x,y)= \sum_{a \in \{0,1\}} P(a,b|\tilde{x},y),
\end{align*}
where $P(a,b|x,y)$ is the probability that the box, given input bits $x,y$ will produce output bits $a,b$. Does the box violate the non-signaling constraints?
\end{subex}

\begin{subex}
Now imagine we get an upgraded version of the same box. This box takes a bit string $x$ of $n$ bits as input on Alice's side and a single bit $y$ on Bob's side. It outputs a string $r$ to Alice such that $P(r_i=0)=P(r_i=1)=\frac12$ for all $1\leq i \leq n$ and outputs the string $b$ such that $b_i=r_i+x_i \cdot y \mod 2$ for all $1 \leq i \leq n$. You can think of this as the 'string' version of the PR box.

Alice and Bob would like to use this box, and classical communication from Alice to Bob, to design another protocol for some form 1-2 oblivious transfer of strings. That is, Alice has two strings $s_0,s_1$ and Bob has a bit $b$ and at the end of the protocol we would like Bob to hold the string $s_b$ while having no knowledge of the other string. Alice inputs a random string $x$ into the box and receives $r$ which she uses to one-time-pad encode her first message $e_0 := s_0 \oplus r$. The other message is encoded as $e_1 := s_1 \oplus x \oplus r$. Both $e_0$ and $e_1$ are sent to Bob. Explain how Bob can recover $e_y$ after inputting his choice bit $y$ into the upgraded box.

Is this really 1-2 oblivious transfer? 
\textbf{Hint: } can we build bit commitment out of this form of 1-2 oblivious transfer? Why, or why not?
\end{subex}

\end{exercise}






\end{document}