\documentclass[a4paper,10pt,landscape,twocolumn]{scrartcl}

\newcommand{\ket}[1]{| #1 \rangle}
\newcommand{\bra}[1]{\langle #1 |}
\newcommand{\proj}[1]{| #1 \rangle \langle #1 |}
\newcommand{\Tr}{\text{Tr}}

%% Settings
\newcommand\problemset{1}
\newcommand\deadline{Update: Friday June 17th, 2022, 20:00h}
\newif\ifcomments
\commentsfalse % hide comments
%\commentstrue % show comments

% Packages
\usepackage[english]{exercises}
\usepackage{wasysym}
\usepackage{hyperref}
\hypersetup{colorlinks=true, urlcolor = blue, linkcolor = blue}
\usepackage{bbm}
\begin{document}

\homeworkproblems

{\sffamily\noindent
%This week's exercises deal with sets, counting and uniform probabilities.
Please hand in your solutions to these exercises in digital form (typed, or scanned from a neatly hand-written version) through Canvas no later than \textbf{\deadline}.  %Problems marked with a $\bigstar$ are generally a bit harder.
}


\begin{exercise}[Purity]
The purity of a quantum state is defined as $\Tr\rho^2$. Consider a $d$-dimensional quantum state $\rho\in \mathbb{C}^{d\times d}$.
\begin{subex}
What is the maximal value of purity and what class of states achieves this value? Prove your answer.
\end{subex}
\begin{subex}
What is the minimal value of purity, what state achieves this value? Prove your answer.
\end{subex}

\begin{subex}
Any qubit density matrix can be represented by the Bloch vector $\vec{r}$, satisfying $|\vec{r}|\leq 1$. For any quantum state $\tau\in \mathbb{C}^{2\times 2}$ we have that $\tau= \frac{1}{2}\left( \mathbbm{1} + \vec{r}\cdot \vec{\sigma} \right)$, where $\vec{\sigma}=(\sigma_x,\sigma_y,\sigma_z)^T$ is the vector of Pauli matrices. How does the purity of $\tau$ relate to $\vec{r}$?
\end{subex}

\end{exercise}



\begin{exercise}[Parity measurements (Exercise 1.5.1)]
Use a projective measurement to measure the parity, in the Hadamard basis, of
the state $\proj{00}$. Compute the probabilities of obtaining measurement outcomes "even" and
"odd", and the resulting post-measurement states. What would the post-measurement states have
been if you had first measured the qubits individually in the Hadamard basis, and then taken the
parity?
\end{exercise}



\begin{exercise}[A three-player game]
Consider the following three-player game: Alice, Bob, and Charlie each receive one bit ($x$, $y$, and $z$, respectively). They are promised that the parity of the three bits is 1 (i.e., $(x,y,z) \in \{(0,0,1), (0,1,0), (1,0,0), (1,1,1)\}$). Their task is to each output a single bit ($a$, $b$, and $c$), such that $a \oplus b \oplus c = xyz$.


\begin{subex}
Find a classical strategy for Alice, Bob, and Charlie, and prove that it is optimal.
\end{subex}

\begin{subex}
As you might expect, they can do better if they are allowed to share entanglement. Suppose that the players each hold one qubit of the state $\ket{GHZ_3} = \frac{1}{\sqrt{2}}(\ket{000} + \ket{111})$. Find a strategy so that the game is won with certainty.
\\\textbf{Hint:} Their first step should be to change their resource state into $\frac{1}{\sqrt{2}}(\ket{000} - \ket{111})$ if and only if $(x,y,z) = (1,1,1)$.
\end{subex}

\end{exercise}

\begin{exercise}[Relation between min-entropy and ignorance]
	Let $K$ be a classical (key) register, and let $E$ be Eve's quantum register. Prove the following statement for arbitrary classical-quantum states $\rho_{KE}$: Eve is ignorant about $K$ if and only if $H_{min}(K|E) = \log|K|$.
\end{exercise}





\end{document}