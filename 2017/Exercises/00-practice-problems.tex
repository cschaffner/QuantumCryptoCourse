\documentclass[a4paper,10pt,landscape,twocolumn]{scrartcl}

\newcommand{\ket}[1]{| #1 \rangle}
\newcommand{\bra}[1]{\langle #1 |}

%% Settings
\newcommand\problemset{0}
\newcommand\deadline{Friday November 18th, 20:00h}
\newif\ifcomments
\commentsfalse % hide comments
%\commentstrue % show comments

% Packages
\usepackage[english]{exercises}
\usepackage{wasysym}
\usepackage{hyperref}
\hypersetup{colorlinks=true, urlcolor = blue, linkcolor = blue}

\begin{document}

\practiceproblems

{\sffamily\noindent
%This week's exercises deal with sets, counting and uniform probabilities.
You do not have to hand in these exercises, they are for practicing only. %Problems marked with a $\bigstar$ are generally a bit harder.
}

\begin{exercise}[Inner product]
The inner product $\langle v_1 | v_2 \rangle$ is defined as $\langle v_1 || v_2 \rangle = \sum_{i=1}^d a^*_ib_i$. Show that $|\langle v_1 | v_2\rangle|^2 = \langle v_1 | v_2 \rangle\langle v_2 | v_1 \rangle$.
\end{exercise}

\begin{exercise}[Valid quantum state]
Verify that for all $\theta, \varphi \in \mathbb{R}$, $\cos(\theta) \ket{0} + \sin(\theta)e^{i\varphi}\ket{1}$ is a valid qubit state.
\end{exercise}

\begin{exercise}[Changing basis]
Express $\ket1$ in the Hadamard basis. That is, find coefficients $\alpha$ and $\beta$ such that $\ket1 = \alpha \ket+ + \beta \ket-$.
\end{exercise}

\begin{exercise}[Measurement]
Consider the state $\ket\psi = \frac{1}{\sqrt{3}}\ket0 + \sqrt{\frac{2}{3}}\ket1$. What are the probabilities $p_0, p_1$ when we measure $\ket\psi$ in the standard basis? What are the probabilities $p_+, p_-$ when we measure $\ket\psi$ in the Hadamard basis?
\end{exercise}

\begin{exercise}[Pauli operations]
\begin{subex}
Write down the Pauli matrices $X, Y$, and $Z$.
\end{subex}
\begin{subex}
Verify that they are unitary.
\end{subex}
\begin{subex}
What rotations of the Bloch sphere do the Paulis represent?
\end{subex}
\end{exercise}

\begin{exercise}[Density operators]
Which of the following matrices can be interpreted as density operators?
\begin{subex}
\[
\left(
\begin{array}{c c}
\frac{1}{4}& \frac{3}{4}\\
\frac{3}{4}& \frac{3}{4}
\end{array}
\right)
\]
\end{subex}
\begin{subex}
$\frac{1}{3}\ket{u}\bra{u} + \frac{2}{3}\ket{v}\bra{v} + \frac{\sqrt{2}}{3}\ket{u}\bra{v} + \frac{\sqrt{2}}{3}\ket{v}\bra{u}$, where $\ket{u}$ and $\ket{v}$ form an orthonormal basis.
\end{subex}
\begin{subex}
\[
\left(
\begin{array}{c c c}
\frac{1}{2}& 0 & \frac{1}{4}\\
0 & \frac{1}{2} & 0\\
\frac{1}{4} & 0 & 0\\
\end{array}
\right)
\]
\end{subex}
\begin{subex}
\[
\left(
\begin{array}{c c c}
\frac{1}{2}& 0 & \frac{1}{4}\\
0 & \frac{1}{2} & 0\\
\frac{1}{4} & 0 & \frac{1}{4}\\
\end{array}
\right)
\]
\end{subex}
\end{exercise}












\end{document}